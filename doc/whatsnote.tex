\documentclass[twoside]{whatsnote}
\usepackage[mono = false]{libertine}
\usepackage{verbatim,framed,hologo}
\providecommand\cls{\textsf}
\providecommand\pkg{\textsf}
\providecommand\url{\texttt}
\providecommand\env{\texttt}
\DeclareRobustCommand\cs[1]{\texttt{\char`\\#1}}
\usepackage[os = mac]{menukeys}

\coverset
  {
    title    = The \pkg{WhatsNote} Class,
    subtitle = Version 3.0B,
    bioinfo  = Mingyu Xia \url{<xiamyphys@gmail.com>},
    color    = MidnightBlue,
    head     = universe/3,
    logo     = cat/sixpointed stars,
  }

\begin{document}

\maketitle

\chapter { Introduction \& Demo }

\section{Cover configuration}

\begin{framed}
  \begin{verbatim}
  \coverset
    {
      title    = The \pkg{WhatsNote} Class,
      subtitle = Version 3.0A,
      bioinfo  = Mingyu Xia \url{<xiamyphys@gmail.com>},
      color    = MidnightBlue,
      head     = universe/3,
      logo     = cat/sixpointed stars,
    }
  \end{verbatim}
\end{framed}

\section{Global Options}

\subsection{Font}

You can set \keys{\cmdmac~math-font} \keys{\cmdmac~main-font} 
\keys{\cmdmac~sans-font} \keys{\cmdmac~mono-font}.
The \keys{\cmdmac~CJK-font} can be set via \hologo{XeLaTeX} compiler.

\begin{framed}
\begin{verbatim}
  math-font    = STIX Two Math,
  main-font    = { {Libertinus Serif} },
  sans-font    = { {Libertinus Sans} },
  mono-font    = { {Libertinus Mono} },
  CJKmain-font =
    { {Zhuque Fangsong (technical preview)} [AutoFakeBold, AutoFakeSlant] },
  CJKsans-font = { {LXGW WenKai} [BoldFont = *-Medium, AutoFakeSlant] },
  CJKmono-font = { {LXGW WenKai Mono} [BoldFont = *-Medium, AutoFakeSlant] },
\end{verbatim}
\end{framed}

\scratch
\begin{solution}
  \lipsum [ 2 ]
  \result {This is the answer}.
\end{solution}

\subsection{Hideanswer}

The default Boolean value of the key \keys{\cmdmac~hideanswer} is true,
environment \env{solution} and content in the macro \cs{result}
will be hidden when enable this.

\section{Left Text \& Right Figure}

\begin{verbatim}
  \begin {textfig} [ ratio ]
    <Left Content> \textfigsplit <Right Content>
  \end {textfig}
\end{verbatim}

The keys in the environment \env{textfig} accepts a ratio that separate by a colon (\cmd:).

\begin{textfig}[ 3:2 ]
  \begin{problem}[ text-fig layout]
    \lipsum [ 1 ]
    \begin{tasks}(4)
      \task A
      \task B
      \task C
      \task D
    \end{tasks}
  \end{problem}
  \textfigsplit
  \centering
  \includegraphics[ width = .9\linewidth ]{example-image}
\end{textfig}

\begin{remark}
  This is an amsthm environment surrounded by a mdframed box.
  \lipsum [ 2 ]
\end{remark}

\section{Other Preset amsthm environments}

\env{theorem}, \env{lemma}, \env{corollary}, \env{proposition}, 
\env{conjecture}, \env{criterion}, \env{assertion}, \env{definition}, 
\env{condition}, \env{example}, \env{exercise}, \env{algorithm},
\env{question}, \env{axiom}, \env{property}, \env{assumption},
\env{hypothesis}, \env{note}, \env{notation}, \env{claim}, \env{summary}, 
\env{acknowledgment}, \env{case}, \env{conclusion}.

\end{document}